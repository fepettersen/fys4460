\documentclass[a4paper,english, 10pt, twoside]{article}
\usepackage[utf8]{inputenc}
\usepackage[T1]{fontenc}
\usepackage[english]{babel}
\usepackage{epsfig}
\usepackage{graphicx}
\usepackage{caption}
\usepackage{subcaption}
\usepackage{amsfonts, amssymb, amsmath}
\usepackage{listings}
\usepackage{float}
\usepackage[top=2cm, bottom=2cm, left=2cm, right=2cm]{geometry}

%opening
\title{Exam preperations FYS4460}
\author{Fredrik E Pettersen\\ f.e.pettersen@fys.uio.no}
\begin{document}

\maketitle
\newpage
\tableofcontents
\newpage

\section{Molecular-dynamics algorithms}
There are several different possibilities for ``variation'' when doing molecular dynamics. First of all 
one can use different potenitals according to what one would like to model. The simplest beeing the 
Lennard-Jones potential, which only includes two-particle interactions. The Lennard-Jones potential is 
typically pretty good for simulations of noble gasses. If we should want to study another material we will 
therefore need another potential. The weber-Stillinger potential includes both two- and three-particle 
interactions, and can model silicone (Si). Silicone will, in equilibrium, form 4-coordinated tetrahedral 
structures, which the Weber-Stillinger potential reproduces. The VKRE (Vashista, Kalia, Rino, and Ebbsjö) 
potiential is another potential which includes both two and three particle interactions, but this potential 
is made (specificly) to simulate $SiO_2$. The two-body part includes three terms, the Coulomb interaction, 
steric repulsion due to ionic sizes, and a charge-dipole interaction resulting from the large
electronic polarizability of $O_2^-$. Yet another example of a possible potential is the reaxFF potential 
which also includes four-particle interactions. This potential lets us study water molecules in an $SiO_2$ 
matrix.\\
In our Lennard-Jones studies we used the Verlet algorithm for time-integration. It is a simple, yet quite 
good integrator which conserves energy very well. It can be viewed as an advanced Euler-Chromer method, 
where we first update the velocity of each particle at half the timestep, then the position is integrated 
one full timestep using the new velocity. When this is completed for all paritcles we calculate two-body 
forces using the new positions before we use the new forces to fully integrate the velocities.\\
All of the potentials mentiones above decrease in strength with the distance between atoms. For the Lennard-
Jones potential, this is illustrated in figure \ref{LJ_cut}. As we see from this figure, there is for all 
practical purposes no interaction between particles that are a certain length apart. This means that we 
can skip the calculations of forces (which is very expensive) between theese particles. As a practical 
measure we will therefore divide the system we are simulating into cells which are the same size as the 
cutoff length and only calculate forces between particles in neighbouring cells. In this way we are sure 
that no important calculations are left out

\begin{figure}[H]
\centering
\includegraphics[scale=0.75]{plot_potential.png}
\caption{Strength of Lennard-Jones potential as a function of distance between atoms}
\label{LJ_cut}
\end{figure}


\section{Molecular-dynamics in the micro-canonical ensemble}

\section{Molecular-dynamics in the micro-canonical ensemble}

\section{Measuring the diffusion constant in molecular-dynamics simulations}

\section{Measuring the radial distribution function in molecular-dynamics simulations}

\section{Thermostats in molecular-dynamics simulations}

\section{Generating a nano-porous material}

\section{Diffusion in a nano-porous material}

\section{Flow in a nano-porous material}

\section{Algorithms for percolation systems}

\section{Percolation on small lattices}

\section{Cluster number density in 1-d percolation}

\section{Correlation length in 1-d percolation}

\section{Cluster size in 1-d percolation}

\section{Measurment and behavior of $P(p,L)$ and $\Pi(p,L)$}

\section{The cluster number density}

\section{Finite size scaling of $\Pi(p,L)$}

\section{Subsets of the spanning cluster}

\section{Random walks}
% Figure - subfigure template (2 by 2 subfigures)
% \begin{figure}[H]
% \centering
%   \begin{subfigure}[b]{0.48\textwidth}
%     \includegraphics[width=\textwidth]{filename}
%     \caption{small caption}
%     \label{KM:age}
%   \end{subfigure}
%   \begin{subfigure}[b]{0.48\textwidth}
%     \includegraphics[width=\textwidth]{filename.png}
%     \caption{small caption}
%     \label{KM:treat}
%   \end{subfigure}
%   
%   \begin{subfigure}[b]{0.48\textwidth}
%    \includegraphics[width=\textwidth]{filename.png}
%    \caption{small caption}
%    \label{KM:asc}
%   \end{subfigure}
%   \begin{subfigure}[b]{0.48\textwidth}
%    \includegraphics[width=\textwidth]{filename.png}
%    \caption{small caption}
%    \label{KM:sex}
%   \end{subfigure}
%   \caption{some caption}
%   \label{KM}
% \end{figure}


\end{document}

