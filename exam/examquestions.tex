\documentclass[a4paper,english, 10pt, twoside]{article}
\usepackage[utf8]{inputenc}
\usepackage[T1]{fontenc}
\usepackage[english]{babel}
\usepackage{epsfig}
\usepackage{graphicx}
\usepackage{caption}
\usepackage{subcaption}
\usepackage{amsfonts, amssymb, amsmath}
\usepackage{listings}
\usepackage{float}
\usepackage[top=2cm, bottom=2cm, left=2cm, right=2cm]{geometry}

%opening
\title{Exam preperations FYS4460}
\author{Fredrik E Pettersen\\ f.e.pettersen@fys.uio.no}
\begin{document}

\maketitle
\newpage
\tableofcontents
\newpage

\section{Molecular-dynamics algorithms}
There are several different possibilities for ``variation'' when doing molecular dynamics. First of all 
one can use different potenitals according to what one would like to model. The simplest beeing the 
Lennard-Jones potential, which only includes two-particle interactions. The Lennard-Jones potential is 
typically pretty good for simulations of noble gasses. If we should want to study another material we will 
therefore need another potential. The weber-Stillinger potential includes both two- and three-particle 
interactions, and can model silicone (Si). Silicone will, in equilibrium, form 4-coordinated tetrahedral 
structures, which the Weber-Stillinger potential reproduces. The VKRE (Vashista, Kalia, Rino, and Ebbsjö) 
potiential is another potential which includes both two and three particle interactions, but this potential 
is made (specificly) to simulate $SiO_2$. The two-body part includes three terms, the Coulomb interaction, 
steric repulsion due to ionic sizes, and a charge-dipole interaction resulting from the large
electronic polarizability of $O_2^-$. Yet another example of a possible potential is the reaxFF potential 
which also includes four-particle interactions. This potential lets us study water molecules in an $SiO_2$ 
matrix.\\
In our Lennard-Jones studies we used the Verlet algorithm for time-integration. It is a simple, yet quite 
good integrator which conserves energy very well. It can be viewed as an advanced Euler-Chromer method, 
where we first update the velocity of each particle at half the timestep, then the position is integrated 
one full timestep using the new velocity. When this is completed for all paritcles we calculate two-body 
forces using the new positions before we use the new forces to fully integrate the velocities.\\
All of the potentials mentiones above decrease in strength with the distance between atoms. For the Lennard-
Jones potential, this is illustrated in figure \ref{LJ_cut}. As we see from this figure, there is for all 
practical purposes no interaction between particles that are a certain length apart. This means that we 
can skip the calculations of forces (which is very expensive) between theese particles. As a practical 
measure we will therefore divide the system we are simulating into cells which are the same size as the 
cutoff length and only calculate forces between particles in neighbouring cells. In this way we are sure 
that no important calculations are left out, and still gain a significant speedup. In fact, should we do 
all pairwise force calculations we will at best need something like $\frac{n^2}{2}$ FLOPS, where as we 
will only need something like $n$ FLOPS when we use cells.\\
We initialize the system by placing it in a face centered cubic lattice (figure \ref{fcc}) and give each 
atom a random velocity drawn from the uniform distribution. We will of course also need to remove any 
initial drift in the system also by rescaling all velocities by the mean of all velocities. Finally, to 
get around the boundary problems, where any potential we have chosen will strictly speaking be incorrect, 
we implement periodic boundary conditions. That is, two particles on each side of the lattice are neighbouring.

\begin{figure}[H]
\centering
\includegraphics[scale=0.75]{plot_potential.png}
\caption{Strength of Lennard-Jones potential as a function of distance between atoms}
\label{LJ_cut}
\end{figure}
\begin{figure}[H]
\centering
\includegraphics[scale=0.45]{fcc_lattice.png}
\caption{Face centered cubic lattice}
\label{fcc}
\end{figure}

\section{Molecular-dynamics in the micro-canonical ensemble}\label{section_2}
If we leave the simulation to itself after setting it up, the temperature will drop quite dramatically 
and then fluctuate around some steady state solution. This is first off all unwanted, and (possibly) 
unphysical. One way to solve this problem is by introducing a thermostat. In theese simulations we are 
in the microcanonical ensemble. That is, The number of particles, volume, and energy are all fixed. In 
the microcanonical ensemble it can be shown that the temperature is 
\begin{equation}\label{temperature}
T = \frac{2E_k}{3N}
\end{equation}

where the kinetic energy is the sum of the kinetic energy off all the particles. This means that we can 
manipulate the temperature of the system by rescaling the velocity of each particle. There are, as usual, 
a few alternatives on how to do this. The berendsen thermostat lets us simulate the canonical ensemble (poorly) by 
``surrounding'' our system with a heat bath, and rescaling the velocities accordingly. The scaling constant 
is shown in eq \ref{berendsen}.
\begin{equation}\label{berendsen}
 \gamma = \sqrt{1+\frac{\Delta t}{\tau}\left(\frac{T_{bath}}{T}-1\right)}
\end{equation}
Another possibility is the Andersen thermostat which simulates collitions between atoms inside the system and in 
the heat bath. Atoms which collide will gain a new, normally distributed velocity with standard deviation 
$\sqrt{k_B T_{bath}/m}$. For all atoms a random, uniformly distributed number is drawn. If the number is less than 
$\frac{\Delta t}{\tau}$, the atom is assigned a new velocity. A downside of the Andersen thermostat is that 
it disturbs the dynamics of lattice vibrations.

\section{Molecular-dynamics in the micro-canonical ensemble}
The ultimate goal of molecular dynamics simuations is to measure macroscopic quantities such as the pressure, 
permeability, viscosity etc. from the microscopical simulations. This can be achieved via statistical mechanics. 
As mentioned in section \ref{section_2} the temperature can be expressed from the microcanonical ensemble as 
equation \ref{temperature} (Insert mathematical argument).\\
In a similar way we can find the pressure of the entire system as it is expressed in equation \ref{pressure}.
\begin{equation}\label{pressure}
p = \vec{F}\cdot\vec{r} 
\end{equation}
Theese quantities can for example be used to say something about how a system will react to a fluid beeing 
driven through it.

\section{Measuring the diffusion constant in molecular-dynamics simulations}\label{section_4}
We can measure the diffusion constant by measuring the mean square displacement of the particles in the system. 
This will relate to the diffusion constant as 
\begin{equation}\label{einstein}
 \langle r^2\rangle \simeq 2dtD
\end{equation}
where d denotes the dimensionality of the system (in most cases 3), t is directly relatable to the number of 
timesteps that have passed and D is the diffusion constant. One problem by using this relation is that, to 
my knowledge, it has only been derived for particles in a homogenous fluid which is much larger than the average 
size of the particles. Also, one must account for the periodic boundary conditions. This is done by summing the 
distance traveled by each particle at each timestep, and carefully accounting for ``jumping'' from one side of 
the system to the other.\\
Equation \ref{einstein} is also widely used to measure the diffusion constant in random walks computations. 
For example we ca use it to measure the diffusion constant of random walkers on a spanning cluster in percolation 
theory. INSERT SOME MORE STUFF

\section{Measuring the radial distribution function in molecular-dynamics simulations}
??

\section{Thermostats in molecular-dynamics simulations}
As mentioned in section \ref{section_2} the system will be represented by the microcanonical ensemble if we leave it to 
itself. We can adjust the system so that it is represented by the canonical ensemble by ``surrounding'' it with 
a heat bath. This is done by introducing a thermostat which (for example) rescales the velocities of the particles 
so that the temperature of the system matches that of the heat bath. The best thermostat for simulations of the 
canonical ensemble is the Nosè-Hoover thermostat, however this is (apparently) rather difficult to implement.\\
In section \ref{section_2} two other alternatives are described, the Andersen and the Berendsen thermostats.

\section{Generating a nano-porous material}
Generally we generate a nanoporous matrix by first thermalizing a ``normal'' system at some temperature. 
Next, we need some algorithm to make it porous. A typical way of doing this is by removing some of the particles 
in the system by some rule, or at least mark them as stationary. The pores can be specified by, for example, 
drawing spheres of a random radius placed at a random position. Atoms outside the sphere are marked as stationary 
(or the other way around). In more complex simulations one usually heats the system a lot, expands it and rapidly cools 
it again. This will give a more realistic geometry.\\
HOW DO WE CHARACTERIZE THE STRUCTURE OF THE MATERIAL AND THE DYNAMICS OF FLUIDS IN SUCH MATERIALS?

\section{Diffusion in a nano-porous material}
The diffusion constant is measured in the same way as we described in section \ref{section_4}. \\
I guess one could expect the diffusion constant to depend on the porosity of the system. 
It would also make sense if the diffusion constant was different in different directions because the connectivity might depend on direction. This 
should (at least for randomly generated matrices) cancel out on larger scales. \\
Diffusion in larger scale porous media is governed by nonlinear diffusion equations which are immensely complicated 
to model (using upwind differences etc). However, the thing all diffusion processes have in common is reaching a steady 
state after some time. This is also expected from the low density fluid in the nanoporous system.

\section{Flow in a nano-porous material}
We can introduce flow in one direction in the nanoporous material by adding a driving force to all particles, much like a gravitational force.
This should make all particles drift in one direction.\\
HOW CAN YOU CHECK YOUR MODEL????\\
To calculate the viscosity of the model we need the superficial velocity distribution of the fluid. One way to estimate the fluid viscosity is 
by simulating flow through a cylindrical pore of a given radius and with a given driving force we find the viscosity from
\begin{align*}
 u(r) &= \frac{\Delta P}{4\mu L}\left(a^2-r^2\right)\\
 \implies \mu &= \frac{nFa^2}{4u(0)}
\end{align*}
The permeability of the nanoporous material can be measured by...

\section{Algorithms for percolation systems}

\section{Percolation on small lattices}

\section{Cluster number density in 1-d percolation}

\section{Correlation length in 1-d percolation}

\section{Cluster size in 1-d percolation}

\section{Measurment and behavior of $P(p,L)$ and $\Pi(p,L)$}

\section{The cluster number density}

\section{Finite size scaling of $\Pi(p,L)$}

\section{Subsets of the spanning cluster}

\section{Random walks}
% Figure - subfigure template (2 by 2 subfigures)
% \begin{figure}[H]
% \centering
%   \begin{subfigure}[b]{0.48\textwidth}
%     \includegraphics[width=\textwidth]{filename}
%     \caption{small caption}
%     \label{KM:age}
%   \end{subfigure}
%   \begin{subfigure}[b]{0.48\textwidth}
%     \includegraphics[width=\textwidth]{filename.png}
%     \caption{small caption}
%     \label{KM:treat}
%   \end{subfigure}
%   
%   \begin{subfigure}[b]{0.48\textwidth}
%    \includegraphics[width=\textwidth]{filename.png}
%    \caption{small caption}
%    \label{KM:asc}
%   \end{subfigure}
%   \begin{subfigure}[b]{0.48\textwidth}
%    \includegraphics[width=\textwidth]{filename.png}
%    \caption{small caption}
%    \label{KM:sex}
%   \end{subfigure}
%   \caption{some caption}
%   \label{KM}
% \end{figure}


\end{document}

