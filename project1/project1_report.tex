\documentclass[a4paper,english, 10pt, twoside]{article}
\usepackage[utf8]{inputenc}
\usepackage[T1]{fontenc}
\usepackage[english]{babel}
\usepackage{epsfig}
\usepackage{graphicx}
\usepackage{amsfonts, amssymb, amsmath}
\usepackage{listings}
\usepackage{float}	%force figures in place with command \begin{figure}[H]

\usepackage[top=2cm, bottom=2cm, left=2cm, right=2cm]{geometry}



%opening
\title{Project 1, FYS4460}
\author{Fredrik E Pettersen\\ f.e.pettersen@fys.uio.no}

\begin{document}

\maketitle

\newpage
\begin{abstract}
In this project we will look at a simple linear second order differential equation.
\end{abstract}
\tableofcontents
\newpage


\section{About the problem}
The goal of this project is to model systems of Argon atoms using the Lennard-Jones potential. 

\section{The algorithm}
\section{Analytic solution}
\section{Results}
First of all, the initial distribution of Argon is visualized in figure \ref{first}. This 
is a so called face centered cubic lattice, where one cube consists of $5\times5\times5 = 25$ 
Argon atoms.
\begin{figure}[H]
\centering
\includegraphics[scale=0.5]{blalbalbla.png}
\caption{The initial configuration of Argon atoms makes a face-centered cubic lattice }
\label{first}
\end{figure} 

Having placed the atoms in the correct formation, we give each atom an initial random velocity 
from the Bolzmann distribution depending on the temperature. \\
For the temperature $T = 119.8K = 1$ in MD units the velocity distribution has 0 mean and 
standard deviation 1. We can check this for the first resultfile (see figure \ref{distribution}).
\begin{figure}[H]
\centering
\includegraphics[scale=0.5]{}
\caption{Velocity distribution at t=0.}
\label{distribution}
\end{figure} 
\section{Stability and precision}
\section{Final comments}

\appendix
\section{Source code}
\lstinputlisting{system.cpp}
\lstinputlisting{cell.cpp}
\lstinputlisting{particle.cpp}
\end{document}
