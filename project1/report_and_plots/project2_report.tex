\documentclass[a4paper,english, 10pt, twoside]{article}
\usepackage[utf8]{inputenc}
\usepackage[T1]{fontenc}
\usepackage[english]{babel}
\usepackage{epsfig}
\usepackage{graphicx}
\usepackage{amsfonts, amssymb, amsmath}
\usepackage{listings}
\usepackage{float}	%force figures in place with command \begin{figure}[H]

\usepackage[top=2cm, bottom=2cm, left=2cm, right=2cm]{geometry}



%opening
\title{Project 2, FYS4460}
\author{Fredrik E Pettersen\\ f.e.pettersen@fys.uio.no}

\begin{document}

\maketitle

\newpage
% \begin{abstract}
% In this project we will modell a collection of interacting Argon atoms confined in a box.\\
% \end{abstract}
\tableofcontents
\newpage


\section{About the problem}
The goal of this project is to model flow in a nanoporous material. For simplicity we will do this using 
the Lennard-Jones potential from project 1, and fix the positions of some of the Argon atoms.

\section{The algorithm}
The algorithm is in essence the same as in project 1 with some minor modifications. Since some of the 
Argon atoms are supposed to be fixed i space, we only need to calculate forces and possitions for the 
ones that are not fixed. Probably the most efficient way of doing this would be to make two separate lists 
of atoms in the program; one containing movable, and one containing immovable atoms. However implementing 
this would (I suspect) be a very demanding modification to my program from the first project, and so I have 
solved this by testing. If the atom is marked as immovable, no calculations are done. In this way I only 
had to add two lines to the force-part of the existing code, and everything worked as it should.

\section{Points of progress}
First of all we will make a cylinder with a radius of 2nm. This cylinder is shown in figure \ref{cylinder} 
where I have visualized only the stationary atoms.
\begin{figure}[H]
 \centering
 \includegraphics[scale=0.5]{cylinder.png}
 \caption{Cylinder with radius 2nm}
 \label{cylinder}
 \end{figure}
Next we generate a matrix consisting of 20 spherical pores at random positions and with random radii 
ranging from 2nm to 3nm. The porosity of this system is $52.4\%$ which I have measured simply by counting 
the number of atoms within the pores and dividing this by the total number of atoms. Compared to what we 
would expect from 20 non-overlapping spherical pores with uniformly drawn random radii ranging from 2 to 
3nm which makes a volume of between $670nm^3$ and $2250nm^3$ making the porosity somewhere between 
$44.7\%$ and $100\%$ with an expectation value of ($\hat{r} = 2.5$) $\phi = 87\%$.\\
There should be quite a few other ways to measure the porosity of the generated matrix which proably are 
a bit more accurate than my approach. One could, for example, calculate the volume of each atom by drawing 
a line to all neighbouring atoms and at the middle of each line draw a normal plane. The volume of an atom 
is then the volume of the resulting figure, making the porosity of the system the ratio between the volume 
of movable atoms to the immovable ones.

We would now like to study a fluid at half the density. The way I interpret this we would like to keep the 
density of the  ``solid'' the same. The only way I can think of which lets us do this is by removing half 
of the particles which are allowed to move. We do this by looping over the movable atoms and drawing a 
random (uniform) number. If it is less than one half, the atom is removed. This will generalize to more 
factors if it should be of interest.\\
Figure \ref{meansquare} shows $\langle r^2(t)\rangle$ for the low-density fluid in the nanoporous matrix 
without any external factors.
\begin{figure}[H]
 \centering
 \includegraphics[scale=0.7]{nanoporous_meansquare.png}
 \label{meansquare}
\end{figure}

Darcy's law states that
$$
U = {k\over\mu}\left(\nabla P - \rho g\right)
$$
inserting for $\rho = {nm\over V}$ and $n = {N\over V}\implies nV = N$ we obtain
$$
U = {k\over\mu}\left(\nabla P - nmg\right)
$$
From Newtons second law know that ma = F which in our case is $F_x$ and so wa canreplace $\rho g$ by 
$nF_x$ 

\section{Results}

Fluid dynamics tells us that a fluid flowing between two stationary plates (of infinite length) will have a 
velocity distribution going as a parabola. We ca try to recreate this scenario in our molecular dynamics model 
by having our Argon - Lennard Jones fluid flowing through a cylider by a driving force (scaled to 1 in the 
dimensionless units) like a gravitational force. Figure \ref{veldir1} shows a plot where the cross section of 
the cylindrical pore has been transformed onto an r. As we can see, the velocity is largest at the center of the 
cylinder ($r=0$) as we expect. We also notice a decrease in velocity as we approach the edges of the pore.\\
It can be shown that the velocity distribution takes the form 
\begin{equation}\label{veldir_eq}
 u(r) = \frac{\Delta P}{4\mu L}\left(a^2-r^2\right)
\end{equation}

where we have $\Delta P = \rho g \Delta h = \frac{NFL}{V}$ and where we know that $u(r) \propto a^2-r^2$. a 
is our system length constant which is $a = 5.72\sigma$. Solving (\ref{veldir_eq}) for $\mu$ at $r=0$ gives us 
$$
\mu = \frac{nFa^2}{4u(0)} = ???
$$
\begin{figure}[H]
\centering
 \includegraphics[scale=0.6]{project2_velocitydistr_1200.png}
 \caption{Radial distribution of velocity in z direction.}
 \label{veldir1}
\end{figure}

% \begin{figure}[H]
% \centering
%  \includegraphics[scale=0.6]{plot_radial_ncells12_T200K_andersen.png}
%  \caption{Radial distribution of atoms for a liquid phase. Simulation with 6912 particles and Andersen thermostat.}
%  \label{radial_12cells}
% \end{figure}

\section{Stability and precision}

\section{Final comments}

% \appendix
% \section{Source code}
% \lstinputlisting{system.cpp}
% \lstinputlisting{cell.cpp}
% \lstinputlisting{particle.cpp}
\end{document}
