\documentclass[a4paper,english, 10pt, twoside]{article}
\usepackage[utf8]{inputenc}
\usepackage[T1]{fontenc}
\usepackage[english]{babel}
\usepackage{epsfig}
\usepackage{graphicx}
\usepackage{amsfonts, amssymb, amsmath}
\usepackage{listings}
\usepackage{float}
\usepackage[top=2cm, bottom=2cm, left=2cm, right=2cm]{geometry}

%opening
\title{Project 3 - Percolation \\ FYS4460}
\author{Fredrik E Pettersen}

\begin{document}
 
 \maketitle
 \section*{a}
 Starting out with a uniform random matrix which we transform to a porous matrix consisting of
 either filled (1) or empty (0) places we will now have clusters of connected, filled places. 
 The ammount of filled and empty places depends on our chosen porosity for the random matrix. 
 An example of a porous, labelled matrix is shown in figure \ref{perc_martix_example}.\\
We want to find the probability distribution for a site in the porous matrix to be in a spanning cluster. A 
spanning cluster is a cluster of connected sites which makes a path from one side of the matrix
to the other. This can be done by finding the area of the spanning cluster, and divide it by
the area of the entire matrix. We find the spanning cluster by labelling the connected clusters 
and checking if the same labels can be found on each side of the matrix. The resulting ``distribution'' 
is shown in figure \ref{P(p,L)}.
This method lets me calculate at least for any rectangular matrix boundary, though I do not know how periodic
bounday conditions will affect the calculations.

\begin{figure}[H]
\centering
%\includegraphics[scale=0.7]{maxvalues.png}
\label{perc_matrix_example}
\end{figure}

\begin{figure}[H]
\centering
 \includegraphics[scale=0.85]{spanningcluster_L300.png}
 \label{P(p,L)}
\end{figure}

\section*{b}
For $p>p_c$ we have that $P(p,L) \propto (p-p_c)^\beta$ ($p_c = 0.59275$). We can find $\beta$ by doing 
a double-logarithmic plot of our measured $P(p,L)$ versus $(p-p_c)$.


\section{c}
We are now gives a collection of random numbers on the form
\begin{lstlisting}
 z = rand(1e6,1).^(-2);
\end{lstlisting}
and area asked to find the distribution function $f_z(z)$ using that $f_z(z) = \frac{dP(Z>z)}{dz}$ and that 
we know f is on the form $f_z(z)\propto z^\alpha$. We can plot the cumulative distribution function $P(Z>z)$ 
(see figure INSERT REFERENCE) and use this to calculate the distribution function. Since 
\begin{align*}
z^\alpha\propto\frac{dP(Z>z)}{dz} \implies \int z^\alpha dz \propto \int dP(Z>z)\\
\frac{1}{\alpha+1}z^{\alpha+1} \propto P(Z>z) \implies (\alpha +1)\log(z)\propto \log(P(Z>z))
\end{align*}
So we see that we are actually measuring $\alpha +1$ if we do a double logarithmic plot of z and the 
cumulative distribution function. From this plot (figure \ref{loglog_f_z}) we find $\alpha = -3/2$. In figure 
\ref{} we have plotted the measured cumulative distribution function and the modelled $f_z(z) = z^\alpha$. 
We see that there is a good fit between the two graphs.

\begin{figure}[H]
 \centering
 \includegraphics[scale=0.85]{loglog_f_z.png}
 \label{loglog_f_z}
\end{figure}

\end{document}
