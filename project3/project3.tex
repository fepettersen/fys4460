\documentclass[a4paper,english, 10pt, twoside]{article}
\usepackage[utf8]{inputenc}
\usepackage[T1]{fontenc}
\usepackage[english]{babel}
\usepackage{epsfig}
\usepackage{graphicx}
\usepackage{caption}
\usepackage{subcaption}

\usepackage{amsfonts, amssymb, amsmath}
\usepackage{listings}
\usepackage{float}
\usepackage[top=2cm, bottom=2cm, left=2cm, right=2cm]{geometry}

%opening
\title{Project 3 - Percolation \\ FYS4460}
\author{Fredrik E Pettersen}

\begin{document}
 
 \maketitle
 \section*{a}
 Starting out with a uniform random matrix which we transform to a porous matrix consisting of
 either filled (1) or empty (0) places we will now have clusters of connected, filled places. 
 The ammount of filled and empty places depends on our chosen porosity for the random matrix. 
 An example of a porous, labelled matrix is shown in figure \ref{perc_matrix_example}.\\
We want to find the probability distribution for a site in the porous matrix to be in a spanning cluster. A 
spanning cluster is a cluster of connected sites which makes a path from one side of the matrix
to the other. This can be done by finding the area of the spanning cluster, and divide it by
the area of the entire matrix. We find the spanning cluster by labelling the connected clusters 
and checking if the same labels can be found on each side of the matrix. The resulting ``distribution'' 
is shown in figure \ref{P(p,L)}.
This method lets me calculate at least for any rectangular matrix boundary, though I do not know how periodic
bounday conditions will affect the calculations.

\begin{figure}[H]
\centering
\includegraphics[scale=0.85]{oppg_a_percmatrix.png}
\caption{An example of a porous $800\times800$ matrix with porosity $p = 0.6$.}
\label{perc_matrix_example}
\end{figure}

\begin{figure}[H]
\centering
 \includegraphics[scale=0.85]{spanningcluster_L300.png}
 \caption{Probability for a randomly selected site to be a part of the spanningcluster}
 \label{P(p,L)}
\end{figure}

\section*{b}
For $p>p_c$ we have that $P(p,L) \propto (p-p_c)^\beta$ ($p_c = 0.59275$). We can find $\beta$ by doing 
a double-logarithmic plot of our measured $P(p,L)$ versus $(p-p_c)$.


\section*{c}
We are now gives a collection of random numbers on the form
\begin{lstlisting}
 z = rand(1e6,1).^(-2);
\end{lstlisting}
and area asked to find the distribution function $f_z(z)$ using that $f_z(z) = \frac{dP(Z>z)}{dz}$ and that 
we know f is on the form $f_z(z)\propto z^\alpha$. We can plot the cumulative distribution function $P(Z>z)$ 
(see figure \ref{alpha}) and use this to calculate the distribution function. Since 
\begin{align*}
z^\alpha\propto\frac{dP(Z>z)}{dz} \implies \int z^\alpha dz \propto \int dP(Z>z)\\
\frac{1}{\alpha+1}z^{\alpha+1} \propto P(Z>z) \implies (\alpha +1)\log(z)\propto \log(P(Z>z))
\end{align*}
So we see that we are actually measuring $\alpha +1$ if we do a double logarithmic plot of z and the 
cumulative distribution function. From this plot (figure \ref{loglog_f_z}) we find $\alpha = -3/2$. In figure 
\ref{alpha} we have plotted the measured cumulative distribution function and the modelled $f_z(z) = z^\alpha$. 
We see that there is a good fit between the two graphs.

\begin{figure}[H]
 \centering
 \includegraphics[scale=0.85]{loglog_f_z.png}
 \caption{Double logarithmic plot of the cumulative distribution function $P(Z>z)$}
 \label{loglog_f_z}
\end{figure}

\begin{figure}[H]
 \centering
 \includegraphics[scale=0.85]{fitted_alpha.png}
  \caption{The measured cumulative distribution function (blue line) and the calculated model $f(z) = z^\alpha$ (red circles).}
 \label{alpha}
\end{figure}

\section*{d}

There are quite a few ways to do logartihmic binning, some more elaborate than others. In the lecturenotes 
we can find one quite elaborate method (which is probably very good). Another very simple method is to use 
the following command i MatLab
\begin{lstlisting}
 z = floor((1.8)^(0:N));
\end{lstlisting}
The method I have settled on is also a quite simple one
\begin{lstlisting}
 z = floor(exp(linspace(1,log(max(s.area)),N)));
\end{lstlisting}

\section*{e}

We can find the cluster numberdensity $n(s,p)$ as the number of clusters with a certain area relative to the 
total number of clusters in the matrix. This is done for different porosities on matrices of the same size. As 
we can see from figure \ref{clustersize} the cluster numberdensity decreases as a power-law with increasing clustersize.
\begin{figure}[H]
\centering
  \begin{subfigure}[b]{0.48\textwidth}
    \includegraphics[width=\textwidth]{cluster_numberdensity_large_p.png}
    \caption{$p > p_c$}
    \label{clustersize:larger}
  \end{subfigure}
  \begin{subfigure}[b]{0.48\textwidth}
    \includegraphics[width=\textwidth]{cluster_numberdensity_small_p.png}
    \label{clustersize:smaller}
    \caption{$p < p_c$}
  \end{subfigure} 
  \caption{Cluster numberdensity for various porosities.}
  \label{clustersize}
\end{figure}

\section*{f}\label{sect_f}

Doing the same thing for differnt system sizes at the critical porosity will let us estimate the exponent $\tau$ from 
the relation $n(s,p;L) = s^\tau$. Doing the usual loglog-plot will give us a straight line where $\tau$ is the 
inclination of the line. Matlab's polyfit function gives us $\tau \approx -1.84$, which is not a perfect fit to the 
tabulated value of $\tau = 2.05$, but still not to bad. The graph used is shown in figure \ref{tau}

\begin{figure}[H]
 \centering
 \includegraphics[scale=0.85]{cluster_numberdensity2.png}
 \caption{An estimate of $\tau$. The plot does not look exactly like it should, probably a result of a 
 to large binsize in the logarithmic binning used for the clustersizes.}
 \label{tau}
\end{figure}

\section*{g}

Using the dataset from the two previous sections we could estimate the typical clustersize $s_\xi$, however 
there is a lot of noise in this data, and our estimate would be a poor one. A better way of estimating $s_\xi$ 
is to make a data-collapse plot. We know that $s_\xi \sim |p-p_c|^{-1/\sigma}$. If we plot the cluster 
numberdensity for increasing p-values (keeping L constant) and multiply with $s^\tau$ which we found in section 
\ref{sect_f} we should get a data-collapse plot where the curves ``fall onto eachother'' for a good value of $\sigma$. 
The tabulated value is $\sigma = 0.395$, and I found $\sigma \approx 0.4$. Figure \ref{sigma} shows the data-
collapse.

\begin{figure}[H]
 \centering
 \includegraphics[scale=0.55]{sigma_10values.png}
 \caption{Datacollapse}
 \label{sigma}
\end{figure}

\section*{h}

We will now find the ``mass'' of the percolating (spanning) cluster at $p = p_c$ as a function of the system size L. 
In 3D the mass of an object is proportional to its volume, it stands to reason then that a 2D object will have a mass 
proportional to it's area. We will also assume (more specifically) that the mass of the percolating cluster will 
satisfy the following
$$
M(p,L) = L^D P(p,L).
$$
Doing a double logarithmic plot of M as a function of L lets us determind the exponent D. The plot is shown in 
figure \ref{mass_D}, and using matlabs polyfit function we find $D \approx 1.89$ which is the tabulated value.
\begin{figure}[H]
 \centering
 \includegraphics[scale=0.75]{mass.png}
 \caption{Mass of the spanningcluster for increasing systemsize L.}
 \label{mass_D}
\end{figure}

\section*{i}

Defining $\Pi(p,L)$ which is the probability of having a percolating cluster, this should go to a Heaviside 
step-function in $p = p_c$ as the system size goes to infinity. We can of course check this (see figure \ref{Pi}), 
but what is more interesting is to try and estimate another exponent using finite size scaling. Finding the probability 
for $\Pi$ to be equal to some porbability (say $p_{\Pi = x}$) which is simply read of figure \ref{Pi} we have the 
relation 
\begin{equation}\label{blergh}
 p_{\Pi = x_1} - p_{\Pi = x_2} = (C_{x_1} - C_{x_2})L^{-1/\nu}
\end{equation}

\begin{figure}[H]
 \centering
 \includegraphics[scale=0.85]{oppg_i_Pi.png}
 \caption{Probability for there to be a percolating cluster on our matrix, plotted for different systemsizes.}
 \label{Pi}
\end{figure}

\section*{j}

It should come as no surprise then that we can use equation \ref{blergh} to estimate $\nu$ by doing a double 
logarithmic plot of $p_{\Pi = x_1} - p_{\Pi = x_2}$ vs L. This plot is shown in figure \ref{estimate_nu}. 
Using MatLab's polyfit function I got $\nu \approx 1.262626$ compared to the analytical value of $\nu={4\over3}$. 
I have only made a rough estimate using some 10 samples when estimating this. 

\begin{figure}[H]
 \centering
 \includegraphics[scale=0.75]{oppg_j_estimate_nu.png}
 \caption{Estimate of $\nu$ for $x_1 = 0.8$ and $x_2 = 0.3$}
 \label{estimate_nu}
\end{figure}

\section*{k}

Using the data we now have gathered we can make an estimate of the percolating thershold $p_c$. Plotting $p_{\Pi = x}$ 
vs $L^{-1/\nu}$ theese lines will intersect the y-axis in $p_c$ (see figure \ref{estimate_pc}). This figure gives the 
estimate of $p_c$ as $p_c \in [0.5923,0.594]$, which is not bad considering the small ammount of samples I have used.

\begin{figure}[H]
 \centering
 \includegraphics[scale=0.75]{oppg_k_rough.png}
 \caption{Estimate of $p_c$ for $x_1 = 0.8$ and $x_2 = 0.3$}
 \label{estimate_pc}
\end{figure}

\section*{l}

We are given the script ``exwalk'' which uses the function ``walk'' to find the singly connected bonds of a percolating 
cluster. The script finds the percolating cluster of a matrix and starts a walker on this cluster. The walker mooves by 
trying to turn left or right as much as possible. It's goal is to make it to the other side of the matrix. Two walkers 
try walking across the cluster from top to bottom, and from right to left. A singly connected bond is a cluster of places 
on the matrix where both walkers have visited. The singly connected bonds of a randomly chosen matrix is visaluzed in figure 
\ref{output_exwalk}. The script finds theese singly connected bonds by storing the path of each walker, that is it's position 
on the matrix, and how long it stayed at each position in the form of which direction it moved further in. Theese matrices 
will be rather sparse, and so multiplying them will result in a matrix containing the places both walkers have visited. 
Some of theese positions will have values larger than 1, and theese positions are the places where at least one of the 
walkers have lingered.

\begin{figure}[H]
 \centering
 \includegraphics[scale=0.95]{oppg_l_output.png}
 \caption{Various output from the script ``exwalk.m``. The bottom left picture shows the singly connected bonds of the 
 percolating cluster. The top left picture is the percolating cluster, and the top right is the percolating cluster 
 with the singly connected bonds on top.}
 \label{output_exwalk}
\end{figure}

\section*{m}

We can also find the mass of the singly connected bonds as a function of system size from the relation $M_{SC} \propto L^{D_{SC}}$. 
Figure \ref{mass_SC} shows this relation averaged over 500 samples. MatLab's polyfit function gives us $D_{SC}\approx  0.7745$.

\begin{figure}[H]
 \centering
 \includegraphics[scale=0.75]{oppg_m_mass.png}
 \caption{Estimate of $M_{SC}$ as a function of L for $L = 2.^{(3:9)}$. 500 samples used per L.}
 \label{mass_SC}
\end{figure}

\section*{q}

We will now study random walks on the percolating cluster. Again, we are given the important parts of the code (one C 
program which can be used in matlab through mex, and one matlab script which uses the program). The C program returns 
two vectors containing the distance moved in x and y direction. We can find the distance $\langle R^2\rangle$ moved 
by the walker by a simple conversion $\langle R^2\rangle = (x(end)-x(1))^2 + (y(end)-y(1))^2$. And to get more accurate 
results we will average this over 5000 samples. We are interested in how $\langle R^2\rangle$ scales with the number of 
steps the walker carries out for different porosities. The results can be found in figure \ref{rsquared} .\\
Producing a data collapse is not possible...

\begin{figure}[H]
 \centering
 \includegraphics[scale=0.75]{oppg_q_take4_multiple.png}
 \caption{$\langle R^2\rangle$ as a function of the number of steps done by the walker. Double logarithmic axes.}
 \label{rsquared}
\end{figure}

\section*{r}

From figure \ref{rsquared} and figure \ref{xi} we can estimate the correlation length $\xi$. Though it might not 
be very clear from fgure \ref{xi}, $\langle r^2\rangle$ will tend to a constant as N tends to infinity because the 
walker is walking on a cluster of finite size (that is, not a spanning cluster). The correlation length squared, 
$\xi^2$ is found where the graphs of $\langle r^2\rangle$ start to fall off. We read this off figures \ref{rsquared} 
and \ref{xi} as happening at roughly $N = 10 000$, making $\xi^2 \approx 1000$ and $\xi \approx 32$.\\
From figure \ref{rsquared} we can also finf the dimensionality of the walker by using the relation 
$\langle r^2\rangle \propto N^{2/d_w}$ meaning that the incline of the curves in figure \ref{rsquared} is ${2\over d_w}$. 
We find $d_w$ to be somewhere between $2.766$ and $2.986$. This means that the walker spanns more than 2 dimensions.

\begin{figure}[H]
 \centering
 \includegraphics[scale=0.75]{oppg_r.png}
 \caption{$\langle R^2\rangle$ as a function of the number of steps done by the walker.}
 \label{xi}
\end{figure}


\end{document}
