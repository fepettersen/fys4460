\documentclass[a4paper,english, 10pt, twoside]{article}
\usepackage[utf8]{inputenc}
\usepackage[T1]{fontenc}
\usepackage[english]{babel}
\usepackage{epsfig}
\usepackage{graphicx}
\usepackage{caption}
\usepackage{subcaption}
\usepackage{amsfonts, amssymb, amsmath}
\usepackage{listings}
\usepackage{float}
\usepackage[top=2cm, bottom=2cm, left=2cm, right=2cm]{geometry}

%opening
\title{Some title}
\author{Fredrik E Pettersen\\ f.e.pettersen@fys.uio.no}
\begin{document}

\maketitle
% 
% \begin{abstract}
% 
% \end{abstract}

\begin{figure}
\centering
\includegraphics[scale=0.9]{oppg_a_56pictures.png}
\caption{$P(p,L)$ for 14 samples pr system size.}
\end{figure}

\begin{figure}
\centering
\includegraphics[scale=0.9]{oppg_i_estimate_nu.png}
\caption{Estimate of the exponent $\nu$ from finite size scaling of $\Pi(p,L)$}
\end{figure}
% 
% \begin{figure}
% \centering
% \includegraphics[scale=0.8]{oppg_i_estimate_nu.png}
% \caption{Estimate of the exponent $\nu$ from finite size scaling of $\Pi(p,L)$}
% \end{figure}

\begin{figure}
\centering
\includegraphics[scale=0.9]{oppg_i_Pi_56pictures.png}
\caption{$\Pi(p,L)$ using 14 samples per system size}
\end{figure}

\begin{figure}[H]
\centering
  \begin{subfigure}[b]{\textwidth}
    \includegraphics[width=\textwidth]{oppg_i_estimate_pc1.png}
    \caption{}
  \end{subfigure}
  
  \begin{subfigure}[b]{\textwidth}
    \includegraphics[width=\textwidth]{oppg_i_estimate_pc2.png}
    \caption{}
  \end{subfigure}
  \caption{estimate of $p_c$ using finite size scaling of $\Pi(p,L)$ and the estimated $\nu$}
\end{figure}

% \begin{figure}
% \centering
% \includegraphics[scale=0.9]{oppg_i_Pi_56pictures.png}
% \caption{$\Pi(p,L)$ using 14 samples per system size}
% \end{figure}

\end{document}

